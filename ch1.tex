\chapter{Theory of Sets}

\setcounter{section}{3} % Starts section counting at 4; change this if you're inserting earlier sections
\section{Indexed Families of Sets}

\textbf{Page 7:} Show that $\bigcup_{\alpha \in \emptyset} A_\alpha = \emptyset$ and that $\bigcap_{\alpha \in \emptyset} A_\alpha = S$.

First, we need to define quantification ($\forall$ and $\exists$) over the empty set:

\begin{lemma} \label{lem:vacuousness}
	For any proposition $P(x)$, $\exists x \in \emptyset, P(x)$ is false and $\forall x \in \emptyset, P(x)$ is true.
\end{lemma}

\begin{proof}[Proof of Lemma~\ref{lem:vacuousness}]
	Let $P(x)$ be some proposition defined for a set $D$.
	If $\exists x \in D, P(x)$, by definition we know there is some $x \in D$ where $P(x)$ holds.
	Therefore, $\exists x \in \emptyset, P(x)$ is false, because for any $x$ where $P(x)$ holds, by definition of $\emptyset$, $x \notin \emptyset$.

	In other words, $\exists x \in D, P(x)$ means we can take an element from $D$ and show $P(x)$ holds; if there is nothing to take from $D$,
	then there is no way to show that $P(x)$ holds!

	We derive that $\forall x \in \emptyset, P(x)$ is true from the previous result and the rules of logic:

	\begin{align*}
		\forall x \in \emptyset, P(x) &\equiv \sim \sim (\forall x \in \emptyset, P(x)) & \text{(double negation)}\\
		&\equiv \sim (\exists x \in \emptyset, \sim P(x)) & \text{(negation of $\forall$)}\\
		&\equiv \sim \text{False} & \text{(vacuous falsehood)}\\
		&\equiv \text{True}
	\end{align*}

	So for logic to be consistent, $\forall x \in \emptyset, P(x)$ must necessarily be true.
\end{proof}

Note: These properties directly correspond to the base cases in the \texttt{all} and \texttt{any} functions provided in many modern programming languages
(including Python, Ruby, Haskell, and C++). For example, consider the following (simplified) implementations of the Haskell functions:

\begin{verbatim}
any :: [Bool] -> Bool
any []       = False
any (x : xs) = x || (any xs)

all :: [Bool] -> Bool
all []       = True
all (x : xs) = x && (all xs)
\end{verbatim}

From this lemma, we can show the given properties of the empty index set:

\begin{proof}
	To say $a \in \bigcup_{\alpha \in I} A_\alpha$ is equivalent to stating $\exists \alpha \in I$ where $a \in A_\alpha$.
	By Lemma~\ref{lem:vacuousness}, $\exists \alpha \in \emptyset, a \in A_\alpha$ is false, regardless of our choice of $a$.

	Therefore, $\bigcup_{\alpha \in \emptyset} A_\alpha = \emptyset$.

	Likewise, to say $a \in \bigcap_{\alpha \in I} A_\alpha$ is equivalent to stating $\forall \alpha \in I, a \in A_\alpha$.
	By Lemma~\ref{lem:vacuousness}, $\forall \alpha \in \emptyset, a \in A_\alpha$ is true, regardless of our choice of $a$.
	In other words, $\forall a \in S, \forall \alpha \in \emptyset, a \in A_\alpha$ is true.

	Therefore, $\bigcap_{\alpha \in \emptyset} A_\alpha = S$.
\end{proof}

\begin{problem}[2]
	Let $A, B, D \subset S$ then \\
	\begin{enumerate}[(a)]
		\item $A \cap (B \cup D) = (A \cap B) \cup (A \cap D)$
		\item $A \cup (B \cap D) = (A \cup B) \cap (A \cup D)$
	\end{enumerate}
\end{problem}

\begin{proof}[Proof of (a)]
	If $A \cap (B \cup D)$ is true then by definition of $\cap$, $x \in A$ and $x \in (B \cup D)$. Therefore $x$ must be in $A$ and $x$ must be in either
	be in $B$ or in $D$ but does not need to be in both $B$ and $D$. So $x$ can be either be in $A \cap B$ or in $A \cap D$, in other words $ x \in (A \cap B) \cup (A \cap D)$.
	Thus by the definition of $\subset$, $A \cap (B \cup D) \subset (A \cap B) \cup (A \cap D)$

	Conversely, if $x \in (A \cap B) \cup (A \cap D)$ then by the definition of $\cup$, $x \in A \cap B$ or $x \in A \cap D$. For either possibility $x \in A$ and $x$ is in
	$B$ or $D$ but not necessarily both. Therefore $x \in A$ and $x \in (B \cup D)$, which is equivalent to $x \in A \cap (B \cup D)$. Thus by definition of $\subset$,
	$(A \cap B) \cup (A \cap D) \subset A \cap (B \cup D)$

	Therefore $A \cap (B \cup D) = (A \cap B) \cup (A \cap D)$
\end{proof}

\begin{proof}[Proof of (b)]
	If $x \in A \cup (B \cap D)$ then by the definition of $\cup$, $x \in A$ or $x \in (B \cap D)$. Thus $x \notin \neg(B \cap D)$, which, according to Demorgan's Law,
	is equal to $ x \not in \neg(B) \cup \neg(D)$. Therefore $x \in 
\end{proof}
\begin{problem}[3]
	Let $\{A_\alpha\}_{\alpha \in I}$ be an indexed family of subsets of a set $S$. Let $J \subset I$. Prove that
	\begin{enumerate}[(a)]
		\item $\bigcap_{\alpha \in J} A_\alpha \supset \bigcap_{\alpha \in I} A_\alpha$
		\item $\bigcup_{\alpha \in J} A_\alpha \subset \bigcup_{\alpha \in I} A_\alpha$
	\end{enumerate}
\end{problem}

% TODO proof of (b)

\begin{proof}[Proof of (a)]
	By definition of $\bigcap$, we know that for all $a \in \bigcap_{\alpha \in I} A_\alpha$, $a \in A_\alpha, \forall \alpha \in I$.
	Since $J \subset I$, we can further conclude that$a \in A_\alpha, \forall \alpha \in J$.

	But, we also know that for all $a \in \bigcap_{\alpha \in J} A_\alpha$, $a \in A_\alpha, \forall \alpha \in J$, again by definition of $\bigcap$.
	Thus, each $a \in \bigcap_{\alpha \in J} A_\alpha$ is also in $\bigcap_{\alpha \in I} A_\alpha$.

	Therefore, $\bigcap_{\alpha \in J} A_\alpha \supset \bigcap_{\alpha \in I} A_\alpha$.
\end{proof}

Note, however, that we do not necessarily have $\bigcap_{\alpha \in J} A_\alpha = \bigcap_{\alpha \in I} A_\alpha$ in the case that $J \neq I$.
It is possible for there to be $a \in \bigcap_{\alpha \in J} A_\alpha$ but for there to exist $A_\alpha$ where $\alpha \in I - J$ that does not include $a$.

\begin{problem}[5]
	Let $I$ be the set of real numbers greater than 0. For each $x \in I$, let $A_x$ be the open interval $(0,x)$.
	Prove that $\bigcap_{x \in I} A_x = \emptyset$ and $\bigcup_{x \in I} A_x = I$.
	For each $x \in I$, let $B_x$ be the closed interval $[0,x]$.
	Prove that $\bigcap_{x \in I} B_x = \{0\}$ and $\bigcup_{x \in I} B_x = I \cup \{0\}$.
\end{problem}

% TODO Proofs for B_x

\begin{proof}[Proof of $\bigcap_{x \in I} A_x = \emptyset$]
	\textit{(By contradiction.)}
	Suppose that there exists $y \in \bigcap_{x \in I} A_x$.
	Thus, for every $x \in I$, we have $y \in A_x$.
	Therefore, we can conclude that $y > 0$ and thus $y \in I$.

	By definition of $A_y$, we know $y \notin A_y$.
	Therefore, $y \notin \bigcap_{x \in I} A_x$ \contradiction.

	Therefore, $\bigcap_{x \in I} A_x = \emptyset$.
\end{proof}

\begin{proof}[Proof of $\bigcup_{x \in I} A_x = I$]
	($\supset$) For every $y \in I$, there exists $x' \in I, x' > y$. Thus, $y \in A_{x'}$.
	Therefore, $y \in \bigcup_{x \in I} A_x$.

	($\subset$) For every $y \in \bigcup_{x \in I} A_x$, there exists $x' \in I$ such that $y \in A_{x'}$.
	By definition of $A_{x'}$, we know that $0 < y < x'$.
	Therefore, $y \in I$.
\end{proof}



\setcounter{section}{5}
\section{Functions}
\begin{problem}[5]
	Let $A$ and $B$ be sets.
	The correspondence $p_1$ that associates with each element $(a,b) \in A \times B$ the element $p_1(a,b) = a$ is a function called \textit{the first projection}.
	The correspondence $p_2$ that associates with each element $(a,b) \in A \times B$ the element $p_2(a,b) = b$ is a function called \textit{the second projection}.

	Prove that if $B \neq \emptyset$, then $p_1 : A \times B \mapsto A$ is onto and if $A \neq \emptyset$, then $p_2 : A \times B \mapsto B$ is onto.
	Under what circumstances is $p_1$ or $p_2$ one--one?
	What is $p_1^{-1}(\{a\})$ for an element $a \in A$?
\end{problem}

% TODO characterize p^-1

\begin{proof}[Proof that $p_1 : A \times B \mapsto A$ is onto if $B \neq \emptyset$]
	Take arbitrary $a \in A$.
	Since $B$ is nonempty, let $b \in B$ be some element of $B$.
	Thus, $(a,b) \in A \times B$.

	$a = p_1(a,b)$.

	Therefore, $p_1$ is onto.
\end{proof}
The proof that $p_2$ is onto is similar.

% NOTE: I'm iffy on the \emptyset stuff in this part. --NJ
\begin{proof}[Proof that $p_1 : A \times B \mapsto A$ is one--one iff $B = \{b\}$ or $B = \emptyset$]
	($\Rightarrow$) \textit{(By contradiction.)}
	Suppose that $B$ has at least two distinct elements. Let $b, b' \in B$ be distinct elements of $B$.

	Since $p_1$ is one--one and $p_1(a,b) = p_1(a,b')$, then $b = b'$. \contradiction

	Therefore, if $p_1$ is one--one, then $B = \{b\}$ or $B = \emptyset$.

	($\Leftarrow$) \textit{Case $B = \emptyset$}:
	If $B = \emptyset$, then $A \times B = \emptyset$, so $p_1 : \emptyset \mapsto A$.
	Therefore, $p_1$ is trivially one--one.

	\textit{Case $B = \{b\}$}:
	Let $a,a' \in A$, $b,b' \in B$ such that $p_1(a,b) = p_1(a',b') = a$.

	By definition of $p_1$, $a = a'$. By definition of $B$, $b = b'$.
	Thus, $(a,b) = (a',b')$.

	Therefore, $p_1$ is one--one.
\end{proof}

\begin{problem}
	Let $A$ and $B$ be sets, with $B \neq \emptyset$.
	For each $b \in B$ the corespondence $j_b$ that associates with each element $a \in A$ the element $j_b(a) = (a,b) \in A \times B$ is a function.

	Prove that for each $b \in B, j_b : A \mapsto A \times B$ is one--one.
	What is $j_b^{-1}(W)$ for a subset $W \subset A \times B$?
\end{problem}

\begin{proof}[Proof that $j_b : A \mapsto A \times B$ is one--one]
	Let $a,a' \in A$ such that $j_b(a) = j_b(a')$.
	Then $(a,b) = (a',b)$, so $a = a'$.
	Therefore, $j_b$ is one--one.
\end{proof}

For a subset $W \subset A \times B$, $j_b^{-1}(W) = \{a \mid (a,b) \in W \}$.
In other words, it is the set of elements of $A$ which appear in $W$ in a pair with $b$.

\begin{problem}[8]
	Let $A$ be a set to which there belongs precisely $n$ distinct objects.
	Prove that there are precisely $2^n$ distinct objects that belong to $2^A$.
\end{problem}

\begin{lemma} \label{lem:binary_string}
	Suppose $A = \{a_1,\cdots,a_n\}$ is a set with $n$ distinct elements.
	Let $b : 2^A \mapsto \{0,1\}^n$ be a function where $b(A')$ is an $n$-bit binary string where the $i^{th}$ bit is $1$ if $a_i \in A'$ and $0$ otherwise.
	Then $b$ is a bijection from $2^A$ to $\{0,1\}^n$.
\end{lemma}

For an example of $b$, if $n = 5$ and $A' = \{a_1,a_3,a_4\}$, then $b(A') = 01101$. % Is it clear that we count right-to-left?

\begin{proof}[Proof of Lemma~\ref{lem:binary_string}]
	(\textit{One--one})
	Let $A', A'' \in 2^A$ such that $b(A') = b(A'') = s$.

	Take arbitrary $a_i \in A'$.
	Then the $i^{th}$ bit of $s$ is set, so $a_i \in A''$ by definition of $b$.
	Similarly, every element of $A''$ is in $A'$.
	Therefore, $A' = A''$ and $b$ is one--one.

	(\textit{Onto})
	Take arbitrary $s$ in $\{0,1\}^n$.
	Let $A' \in 2^{A}$ be the set of $a_i \in A$ where the $i^{th}$ bit of $s$ is set.
	Then $b(A') = s$.
	Therefore, $b$ is onto.
\end{proof}

\begin{proof}[Proof of Problem 8]
	By Lemma~\ref{lem:binary_string}, we have a bijection between $2^A$ and $\{0,1\}^n$.
	Thus it suffices to show that $\{0,1\}^n$ has $2^n$ distinct elements.

	For each bit of an $n$-bit binary string there are two choices: either the bit is 0 or it is 1.
	These choices are independent (the value of one bit does not restrict the value of any other).
	Therefore, there are $2^n$ possible ways to construct a $n$-bit binary string.
\end{proof}
