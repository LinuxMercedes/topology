\chapter{Theory of Sets}

\setcounter{section}{3} % Starts section counting at 4; change this if you're inserting earlier sections
\section{Indexed Families of Sets}

\begin{problem}[3]
	Let $\{A_\alpha\}_{\alpha \in I}$ be an indexed family of subsets of a set $S$. Let $J \subset I$. Prove that
	\begin{enumerate}[(a)]
		\item $\bigcap_{\alpha \in J} A_\alpha \supset \bigcap_{\alpha \in I} A_\alpha$
		\item $\bigcup_{\alpha \in J} A_\alpha \subset \bigcup_{\alpha \in I} A_\alpha$
	\end{enumerate}
\end{problem}

% TODO proof of (b)

\begin{proof}[Proof of (a)]
	By definition of $\bigcap$, we know that for all $a \in \bigcap_{\alpha \in I} A_\alpha$, $a \in A_\alpha, \forall \alpha \in I$.
	Since $J \subset I$, we can further conclude that$a \in A_\alpha, \forall \alpha \in J$.

	But, we also know that for all $a \in \bigcap_{\alpha \in J} A_\alpha$, $a \in A_\alpha, \forall \alpha \in J$, again by definition of $\bigcap$.
	Thus, each $a \in \bigcap_{\alpha \in J} A_\alpha$ is also in $\bigcap_{\alpha \in I} A_\alpha$.

	Therefore, $\bigcap_{\alpha \in J} A_\alpha \supset \bigcap_{\alpha \in I} A_\alpha$.
\end{proof}

Note, however, that we do not necessarily have $\bigcap_{\alpha \in J} A_\alpha = \bigcap_{\alpha \in I} A_\alpha$ in the case that $J \neq I$.
It is possible for there to be $a \in \bigcap_{\alpha \in J} A_\alpha$ but for there to exist $A_\alpha$ where $\alpha \in I - J$ that does not include $a$.

\begin{problem}[5]
	Let $I$ be the set of real numbers greater than 0. For each $x \in I$, let $A_x$ be the open interval $(0,x)$.
	Prove that $\bigcap_{x \in I} A_x = \emptyset$ and $\bigcup_{x \in I} A_x = I$.
	For each $x \in I$, let $B_x$ be the closed interval $[0,x]$.
	Prove that $\bigcap_{x \in I} B_x = \{0\}$ and $\bigcup_{x \in I} B_x = I \cup \{0\}$.
\end{problem}

% TODO Proofs for B_x

\begin{proof}[Proof of $\bigcap_{x \in I} A_x = \emptyset$]
	\textit{(By contradiction.)}
	Suppose that there exists $y \in \bigcap_{x \in I} A_x$.
	Thus, for every $x \in I$, we have $y \in A_x$.
	Therefore, we can conclude that $y > 0$ and thus $y \in I$.

	By definition of $A_y$, we know $y \notin A_y$.
	Therefore, $y \notin \bigcap_{x \in I} A_x$ \contradiction.

	Therefore, $\bigcap_{x \in I} A_x = \emptyset$.
\end{proof}

\begin{proof}[Proof of $\bigcup_{x \in I} A_x = I$]
	($\supset$) For every $y \in I$, there exists $x' \in I, x' > y$. Thus, $y \in A_{x'}$.
	Therefore, $y \in \bigcup_{x \in I} A_x$.

	($\subset$) For every $y \in \bigcup_{x \in I} A_x$, there exists $x' \in I$ such that $y \in A_{x'}$.
	By definition of $A_{x'}$, we know that $0 < y < x'$.
	Therefore, $y \in I$.
\end{proof}



\setcounter{section}{5}
\section{Functions}
\begin{problem}[5]
	Let $A$ and $B$ be sets.
	The correspondence $p_1$ that associates with each element $(a,b) \in A \times B$ the element $p_1(a,b) = a$ is a function called \textit{the first projection}.
	The correspondence $p_2$ that associates with each element $(a,b) \in A \times B$ the element $p_2(a,b) = b$ is a function called \textit{the second projection}.

	Prove that if $B \neq \emptyset$, then $p_1 : A \times B \mapsto A$ is onto and if $A \neq \emptyset$, then $p_2 : A \times B \mapsto B$ is onto.
	Under what circumstances is $p_1$ or $p_2$ one--one?
	What is $p_1^{-1}(\{a\})$ for an element $a \in A$?
\end{problem}

% TODO characterize p^-1

\begin{proof}[Proof that $p_1 : A \times B \mapsto A$ is onto if $B \neq \emptyset$]
	Take arbitrary $a \in A$.
	Since $B$ is nonempty, let $b \in B$ be some element of $B$.
	Thus, $(a,b) \in A \times B$.

	$a = p_1(a,b)$.

	Therefore, $p_1$ is onto.
\end{proof}
The proof that $p_2$ is onto is similar.

% NOTE: I'm iffy on the \emptyset stuff in this part. --NJ
\begin{proof}[Proof that $p_1 : A \times B \mapsto A$ is one--one iff $B = \{b\}$ or $B = \emptyset$]
	($\Rightarrow$) \textit{(By contradiction.)}
	Suppose that $B$ has at least two distinct elements. Let $b, b' \in B$ be distinct elements of $B$.

	Since $p_1$ is one--one and $p_1(a,b) = p_1(a,b')$, then $b = b'$. \contradiction

	Therefore, if $p_1$ is one--one, then $B = \{b\}$ or $B = \emptyset$.

	($\Leftarrow$) \textit{Case $B = \emptyset$}:
	If $B = \emptyset$, then $A \times B = \emptyset$, so $p_1 : \emptyset \mapsto A$.
	Therefore, $p_1$ is trivially one--one.

	\textit{Case $B = \{b\}$}:
	Let $a,a' \in A$, $b,b' \in B$ such that $p_1(a,b) = p_1(a',b') = a$.

	By definition of $p_1$, $a = a'$. By definition of $B$, $b = b'$.
	Thus, $(a,b) = (a',b')$.

	Therefore, $p_1$ is one--one.
\end{proof}

\begin{problem}
	Let $A$ and $B$ be sets, with $B \neq \emptyset$.
	For each $b \in B$ the corespondence $j_b$ that associates with each element $a \in A$ the element $j_b(a) = (a,b) \in A \times B$ is a function.

	Prove that for each $b \in B, j_b : A \mapsto A \times B$ is one--one.
	What is $j_b^{-1}(W)$ for a subset $W \subset A \times B$?
\end{problem}

\begin{proof}[Proof that $j_b : A \mapsto A \times B$ is one--one]
	Let $a,a' \in A$ such that $j_b(a) = j_b(a')$.
	Then $(a,b) = (a',b)$, so $a = a'$.
	Therefore, $j_b$ is one--one.
\end{proof}

For a subset $W \subset A \times B$, $j_b^{-1}(W) = \{a | (a,b) \in W \}$.
In other words, it is the set of elements of $A$ which appear in a pair with $b$ in $W$.
